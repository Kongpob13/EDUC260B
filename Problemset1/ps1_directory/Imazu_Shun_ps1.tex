% Options for packages loaded elsewhere
\PassOptionsToPackage{unicode}{hyperref}
\PassOptionsToPackage{hyphens}{url}
\PassOptionsToPackage{dvipsnames,svgnames,x11names}{xcolor}
%
\documentclass[
]{article}
\usepackage{amsmath,amssymb}
\usepackage{lmodern}
\usepackage{iftex}
\ifPDFTeX
  \usepackage[T1]{fontenc}
  \usepackage[utf8]{inputenc}
  \usepackage{textcomp} % provide euro and other symbols
\else % if luatex or xetex
  \usepackage{unicode-math}
  \defaultfontfeatures{Scale=MatchLowercase}
  \defaultfontfeatures[\rmfamily]{Ligatures=TeX,Scale=1}
\fi
% Use upquote if available, for straight quotes in verbatim environments
\IfFileExists{upquote.sty}{\usepackage{upquote}}{}
\IfFileExists{microtype.sty}{% use microtype if available
  \usepackage[]{microtype}
  \UseMicrotypeSet[protrusion]{basicmath} % disable protrusion for tt fonts
}{}
\makeatletter
\@ifundefined{KOMAClassName}{% if non-KOMA class
  \IfFileExists{parskip.sty}{%
    \usepackage{parskip}
  }{% else
    \setlength{\parindent}{0pt}
    \setlength{\parskip}{6pt plus 2pt minus 1pt}}
}{% if KOMA class
  \KOMAoptions{parskip=half}}
\makeatother
\usepackage{xcolor}
\usepackage[margin=1in]{geometry}
\usepackage{color}
\usepackage{fancyvrb}
\newcommand{\VerbBar}{|}
\newcommand{\VERB}{\Verb[commandchars=\\\{\}]}
\DefineVerbatimEnvironment{Highlighting}{Verbatim}{commandchars=\\\{\}}
% Add ',fontsize=\small' for more characters per line
\usepackage{framed}
\definecolor{shadecolor}{RGB}{248,248,248}
\newenvironment{Shaded}{\begin{snugshade}}{\end{snugshade}}
\newcommand{\AlertTok}[1]{\textcolor[rgb]{0.94,0.16,0.16}{#1}}
\newcommand{\AnnotationTok}[1]{\textcolor[rgb]{0.56,0.35,0.01}{\textbf{\textit{#1}}}}
\newcommand{\AttributeTok}[1]{\textcolor[rgb]{0.77,0.63,0.00}{#1}}
\newcommand{\BaseNTok}[1]{\textcolor[rgb]{0.00,0.00,0.81}{#1}}
\newcommand{\BuiltInTok}[1]{#1}
\newcommand{\CharTok}[1]{\textcolor[rgb]{0.31,0.60,0.02}{#1}}
\newcommand{\CommentTok}[1]{\textcolor[rgb]{0.56,0.35,0.01}{\textit{#1}}}
\newcommand{\CommentVarTok}[1]{\textcolor[rgb]{0.56,0.35,0.01}{\textbf{\textit{#1}}}}
\newcommand{\ConstantTok}[1]{\textcolor[rgb]{0.00,0.00,0.00}{#1}}
\newcommand{\ControlFlowTok}[1]{\textcolor[rgb]{0.13,0.29,0.53}{\textbf{#1}}}
\newcommand{\DataTypeTok}[1]{\textcolor[rgb]{0.13,0.29,0.53}{#1}}
\newcommand{\DecValTok}[1]{\textcolor[rgb]{0.00,0.00,0.81}{#1}}
\newcommand{\DocumentationTok}[1]{\textcolor[rgb]{0.56,0.35,0.01}{\textbf{\textit{#1}}}}
\newcommand{\ErrorTok}[1]{\textcolor[rgb]{0.64,0.00,0.00}{\textbf{#1}}}
\newcommand{\ExtensionTok}[1]{#1}
\newcommand{\FloatTok}[1]{\textcolor[rgb]{0.00,0.00,0.81}{#1}}
\newcommand{\FunctionTok}[1]{\textcolor[rgb]{0.00,0.00,0.00}{#1}}
\newcommand{\ImportTok}[1]{#1}
\newcommand{\InformationTok}[1]{\textcolor[rgb]{0.56,0.35,0.01}{\textbf{\textit{#1}}}}
\newcommand{\KeywordTok}[1]{\textcolor[rgb]{0.13,0.29,0.53}{\textbf{#1}}}
\newcommand{\NormalTok}[1]{#1}
\newcommand{\OperatorTok}[1]{\textcolor[rgb]{0.81,0.36,0.00}{\textbf{#1}}}
\newcommand{\OtherTok}[1]{\textcolor[rgb]{0.56,0.35,0.01}{#1}}
\newcommand{\PreprocessorTok}[1]{\textcolor[rgb]{0.56,0.35,0.01}{\textit{#1}}}
\newcommand{\RegionMarkerTok}[1]{#1}
\newcommand{\SpecialCharTok}[1]{\textcolor[rgb]{0.00,0.00,0.00}{#1}}
\newcommand{\SpecialStringTok}[1]{\textcolor[rgb]{0.31,0.60,0.02}{#1}}
\newcommand{\StringTok}[1]{\textcolor[rgb]{0.31,0.60,0.02}{#1}}
\newcommand{\VariableTok}[1]{\textcolor[rgb]{0.00,0.00,0.00}{#1}}
\newcommand{\VerbatimStringTok}[1]{\textcolor[rgb]{0.31,0.60,0.02}{#1}}
\newcommand{\WarningTok}[1]{\textcolor[rgb]{0.56,0.35,0.01}{\textbf{\textit{#1}}}}
\usepackage{graphicx}
\makeatletter
\def\maxwidth{\ifdim\Gin@nat@width>\linewidth\linewidth\else\Gin@nat@width\fi}
\def\maxheight{\ifdim\Gin@nat@height>\textheight\textheight\else\Gin@nat@height\fi}
\makeatother
% Scale images if necessary, so that they will not overflow the page
% margins by default, and it is still possible to overwrite the defaults
% using explicit options in \includegraphics[width, height, ...]{}
\setkeys{Gin}{width=\maxwidth,height=\maxheight,keepaspectratio}
% Set default figure placement to htbp
\makeatletter
\def\fps@figure{htbp}
\makeatother
\setlength{\emergencystretch}{3em} % prevent overfull lines
\providecommand{\tightlist}{%
  \setlength{\itemsep}{0pt}\setlength{\parskip}{0pt}}
\setcounter{secnumdepth}{-\maxdimen} % remove section numbering
\ifLuaTeX
  \usepackage{selnolig}  % disable illegal ligatures
\fi
\IfFileExists{bookmark.sty}{\usepackage{bookmark}}{\usepackage{hyperref}}
\IfFileExists{xurl.sty}{\usepackage{xurl}}{} % add URL line breaks if available
\urlstyle{same} % disable monospaced font for URLs
\hypersetup{
  pdftitle={Problem Set \#1},
  colorlinks=true,
  linkcolor={Maroon},
  filecolor={Maroon},
  citecolor={Blue},
  urlcolor={blue},
  pdfcreator={LaTeX via pandoc}}

\title{Problem Set \#1}
\author{}
\date{\vspace{-2.5em}1/14/2023}

\begin{document}
\maketitle

\textcolor{red}{\textbf{Grade: /18}}

install.packages(``ggplot'') library(ggplot)

\hypertarget{overview}{%
\section{Overview}\label{overview}}

In this problem set, you will practice organizing files and folders for
a new project. You'll also get practice reading and writing data, which
are common tasks you'll likely encounter in a real-life project. From
now on you'll be doing all your work in an \textbf{R script} instead of
an \textbf{R Markdown file}.

\hypertarget{part-i-setting-up-project-directory-and-r-script}{%
\subsection{Part I: Setting up project directory and R
script}\label{part-i-setting-up-project-directory-and-r-script}}

\textcolor{red}{\textbf{/0.5}}

\begin{enumerate}
\def\labelenumi{\arabic{enumi}.}
\item
  You'll create a new \textbf{RStudio project} for this problem set:

  \begin{itemize}
  \tightlist
  \item
    Open up \textbf{RStudio} and on the top right corner of the window,
    select \texttt{New\ Project} under the dropdown menu
  \item
    Select \texttt{New\ Directory} then \texttt{New\ Project} to create
    a new project

    \begin{itemize}
    \tightlist
    \item
      \emph{Note}: Selecting \texttt{New\ Directory} means that the
      folder (i.e., directory) where the R project will live does not
      yet exist; thus, this step will (A) create the directory named
      \texttt{ps1\_directory} and (B) create the file named
      \texttt{ps1\_directory.Rproj} within that directory
    \item
      Alternatively, you may manually create the directory named
      \texttt{ps1\_directory} and then select the
      \texttt{Existing\ Directory} option
    \end{itemize}
  \item
    Fill in the name of your new project directory (e.g.,
    \texttt{ps1\_directory}), browse where you want to create the
    project (e.g., \texttt{Downloads} or \texttt{Documents} folder,
    etc.), and click \texttt{Create\ Project}
  \end{itemize}

  \includegraphics[width=4.27083in,height=\textheight]{rstudio_project.png}
\end{enumerate}

\textcolor{red}{\textbf{/0.5}}

\begin{enumerate}
\def\labelenumi{\arabic{enumi}.}
\setcounter{enumi}{1}
\tightlist
\item
  Download the \textbf{Problem set R script template} available under
  the \textbf{Syllabus \& Resources} section of the
  \href{https://anyone-can-cook.github.io/rclass2/}{class website} (or
  click
  \href{https://anyone-can-cook.github.io/rclass2/assets/resources/ps_template.R}{here}).
  Rename the downloaded \texttt{ps\_template.R} to your
  \texttt{lastname\_firstname\_ps1.R} and save it inside your newly
  created project directory (e.g., \texttt{ps1\_directory}).
\end{enumerate}

\textcolor{red}{\textbf{/1}}

\begin{enumerate}
\def\labelenumi{\arabic{enumi}.}
\setcounter{enumi}{2}
\tightlist
\item
  Open your \texttt{lastname\_firstname\_ps1.R} script in
  \textbf{RStudio} (check the top right corner of the window to make
  sure you have the \textbf{Rstudio project} you created in the first
  question opened). Modify the header information of the R script (e.g.,
  your name, the date, etc.), then load the \texttt{tidyverse},
  \texttt{readxl}, and \texttt{labelled} libraries under the
  \textbf{libraries} section of the given template. Don't forget to run
  your code to load the libraries. (\emph{Hint}: You can run a line of
  code by clicking on that line and hitting
  \texttt{ctrl}+\texttt{enter}/\texttt{cmd}+\texttt{enter} on the
  keyboard)
\end{enumerate}

install.packages(``labelled'')
\#\#-------------------------------------- library(tidyverse)
library(readxl) library(labelled)
\#\#--------------------------------------

\textcolor{red}{\textbf{/1}}

\begin{enumerate}
\def\labelenumi{\arabic{enumi}.}
\setcounter{enumi}{3}
\item
  Next, you'll create a few subdirectories inside your project
  directory. Specifically, you'll create a \texttt{data} folder to
  contain all raw data you'll be working with and an \texttt{analysis}
  folder, containing the \texttt{plots} and \texttt{files} subfolders,
  for storing all your outputs. Your folder structure should look like
  this:

\begin{verbatim}
ps1_directory/
|
|- data/
|- analysis/
   |- plots/
   |- files/
|- lastname_firstname_ps1.R
\end{verbatim}

  You can create these directories in 2 ways: (A) Using the
  \texttt{dir.create()} function in your \textbf{Console} or (B)
  Clicking on the \texttt{New\ Folder} button under the \textbf{Files}
  tab.

  \includegraphics{rstudio_folder.png}

  Once you've created the directories, use \texttt{file.path()} to
  create directory path objects for them under the \textbf{directory
  paths} section of your \texttt{lastname\_firstname\_ps1.R}.
  Specifically, create the following objects (\emph{Hint}: Make sure to
  write your paths relative to your project directory, and don't forget
  to run these lines of code after you're done):

  \begin{itemize}
  \tightlist
  \item
    \texttt{input\_data\_dir}: path to \texttt{data/} folder
  \item
    \texttt{output\_plot\_dir}: path to \texttt{plots/} folder
  \item
    \texttt{output\_file\_dir}: path to \texttt{files/} folder
  \end{itemize}
\end{enumerate}

\#\#-------------------------------------------------------
input\_data\_dir \textless- file.path(``.'', ``data'') output\_plot\_dir
\textless- file.path(``.'', ``analysis'', ``plots'') output\_file\_dir
\textless- file.path(``.'', ``analysis'', ``files'')
\#\#-------------------------------------------------------

\hypertarget{part-ii-exploring-directory-structure-and-reading-data-files}{%
\subsection{Part II: Exploring directory structure and reading data
files}\label{part-ii-exploring-directory-structure-and-reading-data-files}}

\textcolor{red}{\textbf{/1.5}}

\begin{enumerate}
\def\labelenumi{\arabic{enumi}.}
\item
  Now, it's time to start writing code in the main body of your
  \texttt{lastname\_firstname\_ps1.R}. Make sure to add comments in your
  script (i.e., lines starting with \texttt{\#}) to clearly label which
  part/question in the problem set you are answering. For this question,
  your task is to write code to do each of the following:

  \begin{itemize}
  \tightlist
  \item
    Print the absolute filepath of your current working directory
  \item
    List all folders and files in your current working directory
  \item
    List all folders and files inside the \texttt{analysis/} directory
  \end{itemize}

  Run the code and \textbf{copy the outputs to your R script by adding
  them as comments under each respective line of code}.
\end{enumerate}

\#\#--------------------------------------------------- getwd()
\#``C:/Users/imaim/OneDrive/デスクトップ/23-Winter/260-Programing/Week-1/ps1\_directory''
list.files() \#'' analysis'' ``data'' ``Imazu\_Shun\_ps1.Rmd''
``Imazu\_Shun\_ps1.tex'' ``ps1\_directory.Rproj'' list.files(path =
``./analysis'')

\textcolor{red}{\textbf{/2.5 }}

\begin{enumerate}
\def\labelenumi{\arabic{enumi}.}
\setcounter{enumi}{1}
\item
  So far, all our subdirectories have no files in them. Let's change
  that! Add code to your \texttt{lastname\_firstname\_ps1.R} to complete
  the following:

  \begin{itemize}
  \tightlist
  \item
    Download the data from
    \texttt{https://github.com/anyone-can-cook/rclass2/raw/main/data/ps2\_files.zip}
    into your \texttt{input\_data\_dir}
  \item
    Unzip the downloaded zipped folder and make sure the files are
    extracted into \texttt{input\_data\_dir}
  \item
    List the contents of \texttt{input\_data\_dir} and copy the output
    as a comment in your script
  \end{itemize}
\end{enumerate}

\#\#-------------------------------------------------------
download.file(url =
``\url{https://github.com/anyone-can-cook/rclass2/raw/main/data/ps2_files.zip}'',
destfile = file.path(input\_data\_dir, ``ps2\_files.zip''))
unzip(zipfile = file.path(input\_data\_dir, ``ps2\_files.zip''), exdir =
input\_data\_dir) list.files(input\_data\_dir)
\#``college\_scorecard.csv'' ``college\_scorecard\_dict.xlsx''
``ps2\_files.zip''

\textcolor{red}{\textbf{/1.5}}

\begin{enumerate}
\def\labelenumi{\arabic{enumi}.}
\setcounter{enumi}{2}
\item
  Read in the data from \texttt{college\_scorecard.csv} and name the
  object \texttt{college\_scorecard\_df}. Specify the arguments in the
  function you use to read in the data so that the following criteria
  are met:

  \begin{itemize}
  \tightlist
  \item
    Because missing values in the data are represented by
    \texttt{\textquotesingle{}.\textquotesingle{}}, read in the data so
    that these values are replaced with \texttt{NA}
  \item
    Read in the \texttt{control} column so that it is of
    \texttt{integer} type
  \end{itemize}
\end{enumerate}

\#\#------------------------------------------------------
college\_scorecard\_df \textless- read\_csv(file =
file.path(input\_data\_dir, ``college\_scorecard.csv''), col\_types =
cols(control = col\_integer()),na = ``.'')
\#\#------------------------------------------------------

\textcolor{red}{\textbf{/2}}

\begin{enumerate}
\def\labelenumi{\arabic{enumi}.}
\setcounter{enumi}{3}
\item
  Read in the data dictionary from
  \texttt{college\_scorecard\_dict.xlsx} and name the object
  \texttt{college\_scorecard\_dict}. Specify the arguments in the
  function you use to read in the data so that the following criteria
  are met:

  \begin{itemize}
  \tightlist
  \item
    You are reading in the \texttt{institution\_data\_dictionary} sheet
  \item
    Specify the column names to be:
    \texttt{\textquotesingle{}var\_label\textquotesingle{},\ \textquotesingle{}category\textquotesingle{},\ \textquotesingle{}dev\_name\textquotesingle{},\ \textquotesingle{}data\_type\textquotesingle{},\ \textquotesingle{}var\_name\textquotesingle{},\ \textquotesingle{}val\_name\textquotesingle{},\ \textquotesingle{}val\_label\textquotesingle{},\ \textquotesingle{}source\textquotesingle{},\ \textquotesingle{}notes\textquotesingle{}}
  \item
    Since you've specified custom column names, you'll need to skip the
    first row of data you read in (i.e., the heading row)
  \end{itemize}
\end{enumerate}

\#\#--------------------------------------------------------\\
college\_scorecard\_dict \textless- read\_excel(path =
file.path(input\_data\_dir, ``college\_scorecard\_dict.xlsx''), sheet =
``institution\_data\_dictionary'', col\_names = c(`var\_label',
`category', `dev\_name', `data\_type', `var\_name', `val\_name',
`val\_label', `source', `notes'), skip = 1)
\#\#--------------------------------------------------------

\hypertarget{part-iii-data-manipulations-and-writing-data-files}{%
\subsection{Part III: Data manipulations and writing data
files}\label{part-iii-data-manipulations-and-writing-data-files}}

\textcolor{red}{\textbf{/2.5}}

\begin{enumerate}
\def\labelenumi{\arabic{enumi}.}
\item
  There are a lot of variables and information in the data dictionary.
  Let's narrow it down to include just the variables we're interested
  in. To help prepare you for this, first perform the following
  investigations:

  \begin{itemize}
  \tightlist
  \item
    Use \texttt{names()} to print the column names of
    \texttt{college\_scorecard\_df} (i.e., our variables of interest)
  \item
    Use \texttt{str()} to investigate the structure of the
    \texttt{var\_name} column in \texttt{college\_scorecard\_dict}
    (\emph{Hint}: Notice that the variable names in the data dictionary
    are all capitalized)
  \end{itemize}
\end{enumerate}

\#\#------------------------------------------------------------
names(college\_scorecard\_df) \#``instnm'' ``control''
``tuitionfee\_in'' ``tuitionfee\_out'' ``md\_earn\_wne\_p10''
str(college\_scorecard\_dict\$var\_name) \#chr {[}1:2301{]} ``UNITID''
``OPEID'' ``OPEID6'' ``INSTNM'' ``CITY'' ``STABBR'' ``ZIP''
``ACCREDAGENCY'' \ldots{}
\#\#------------------------------------------------------------

\begin{verbatim}
Now, create a new object called `college_scorecard_dict_subset` from `college_scorecard_dict` where you:

- Select only the variables: `var_name, var_label, val_name, val_label`
- Mutate the `var_name` variable so that the variable names contained in this column are all lowercase (_Hint_: Use the `str_to_lower()` function to help)
- Filter the dataframe so that it only contains variables that are in `college_scorecard_df` (_Hint_: Use `names()` to get the variables in `college_scorecard_df`, and use `%in%` to help filter for only those variables)
\end{verbatim}

\#\#----------------------------------------------------------------------
college\_scorecard\_dict\_subset \textless- college\_scorecard\_dict
\%\textgreater\% select(var\_name, var\_label, val\_name, val\_label)
\%\textgreater\% mutate(var\_name = str\_to\_lower(var\_name))
\%\textgreater\% filter(var\_name \%in\% names(college\_scorecard\_df))
\#\#----------------------------------------------------------------------

\textcolor{red}{\textbf{/1}}

\begin{enumerate}
\def\labelenumi{\arabic{enumi}.}
\setcounter{enumi}{1}
\item
  Take a look at the \texttt{college\_scorecard\_dict\_subset} object
  you created in the previous step. You can do this by running
  \texttt{View(college\_scorecard\_dict\_subset)} in the
  \textbf{Console} tab (bottom left panel of R Studio) or clicking on
  the dataframe name under the \textbf{Environment} tab (top right panel
  of R Studio). You should see the data dictionary information for the
  variables in your \texttt{college\_scorecard\_df}.

  Using the information from \texttt{college\_scorecard\_dict\_subset},
  create a new dataframe called
  \texttt{college\_scorecard\_df\_analysis} from
  \texttt{college\_scorecard\_df} where you add the following labels:

  \begin{itemize}
  \tightlist
  \item
    Variable labels for the 5 variables:
    \texttt{instnm,\ control,\ tuitionfee\_in,\ tuitionfee\_out,\ md\_earn\_wne\_p10}
    (\emph{Hint}: Use \texttt{set\_variable\_labels()} to add variable
    labels)
  \item
    Value labels to the \texttt{control} variable (\emph{Hint}: Use
    \texttt{set\_value\_labels()} to add value labels)
  \end{itemize}
\end{enumerate}

\#\#---------------------------------------------------------------------
college\_scorecard\_df\_analysis \textless- college\_scorecard\_df
\%\textgreater\% set\_variable\_labels(instnm = ``Institution name'',
control = ``Control of institution'', tuitionfee\_in = ``In-state
tuition and fees'', tuitionfee\_out = ``Out-of-state tuition and fees'',
md\_earn\_wne\_p10 = ``Median earnings of students working and not
enrolled 10 years after entry'') \%\textgreater\%
set\_value\_labels(control = c(``Public'' = 1, ``Private nonprofit'' =
2, ``Private for-profit'' = 3))
\#\#----------------------------------------------------------------------

\textcolor{red}{\textbf{/1}}

View(college\_scorecard\_dict\_subset)

\begin{enumerate}
\def\labelenumi{\arabic{enumi}.}
\setcounter{enumi}{2}
\tightlist
\item
  Create a new column in \texttt{college\_scorecard\_df\_analysis}
  called \texttt{school\_type} that is \texttt{1} if the institution is
  public and \texttt{2} if it is private. We'll be using this variable
  in the next question when we visualize the data between public and
  private institutions.
\end{enumerate}

\#\#-----------------------------------------------------------------------
college\_scorecard\_df\_analysis
\textless-college\_scorecard\_df\_analysis \%\textgreater\%
mutate(school\_type = recode(control, ``1'' = 1, ``2'' = 2, ``3'' = 2))
\#\#------------------------------------------------------------------------

\textcolor{red}{\textbf{/1}}

\begin{enumerate}
\def\labelenumi{\arabic{enumi}.}
\setcounter{enumi}{3}
\item
  Copy the code provided below to your script and run it. This will
  create the following 2 scatterplots and save the plots as
  \texttt{.png} files in your \texttt{output\_plot\_dir}:

  \begin{itemize}
  \tightlist
  \item
    \texttt{out\_of\_state\_tuition\_earnings.png}: Shows the
    relationship between out-of-state tuition of the institutions and
    the median earnings of their students 10 years after completion
  \item
    \texttt{in\_state\_tuition\_earnings.png}: Shows the relationship
    between in-state tuition of the institutions and the median earnings
    of their students 10 years after completion
  \end{itemize}

  After running the provided code to save the plots, add a line of code
  of your own that lists out the contents of your
  \texttt{output\_plot\_dir}. Copy the output you see into your R script
  as a comment.
\end{enumerate}

\begin{Shaded}
\begin{Highlighting}[]
\CommentTok{\# Plot relationship between out{-}of{-}state tuition and median earnings}
\FunctionTok{png}\NormalTok{(}\FunctionTok{file.path}\NormalTok{(output\_plot\_dir, }\StringTok{\textquotesingle{}out\_of\_state\_tuition\_earnings.png\textquotesingle{}}\NormalTok{))}
\FunctionTok{ggplot}\NormalTok{(college\_scorecard\_df\_analysis, }\FunctionTok{aes}\NormalTok{(}\AttributeTok{x =}\NormalTok{ tuitionfee\_out, }\AttributeTok{y =}\NormalTok{ md\_earn\_wne\_p10, }\AttributeTok{color =} \FunctionTok{as.factor}\NormalTok{(school\_type))) }\SpecialCharTok{+}
  \FunctionTok{geom\_point}\NormalTok{() }\SpecialCharTok{+}
  \FunctionTok{geom\_smooth}\NormalTok{() }\SpecialCharTok{+}
  \FunctionTok{scale\_color\_discrete}\NormalTok{(}\AttributeTok{name =} \StringTok{\textquotesingle{}School Type\textquotesingle{}}\NormalTok{, }\AttributeTok{labels =} \FunctionTok{c}\NormalTok{(}\StringTok{\textquotesingle{}Public\textquotesingle{}}\NormalTok{, }\StringTok{\textquotesingle{}Private\textquotesingle{}}\NormalTok{)) }\SpecialCharTok{+}
  \FunctionTok{xlab}\NormalTok{(}\StringTok{\textquotesingle{}Tuition (Out{-}of{-}State)\textquotesingle{}}\NormalTok{) }\SpecialCharTok{+} \FunctionTok{ylab}\NormalTok{(}\StringTok{\textquotesingle{}Earnings 10 years after completion\textquotesingle{}}\NormalTok{)}
\FunctionTok{dev.off}\NormalTok{()}

\DocumentationTok{\#\#{-}{-}{-}{-}{-}{-}{-}{-}{-}{-}{-}{-}{-}{-}{-}{-}{-}{-}{-}{-}{-}{-}{-}{-}{-}{-}{-}{-}{-}{-}{-}{-}{-}{-}{-}{-}{-}{-}{-}{-}{-}{-}{-}{-}{-}{-}{-}{-}{-}{-}{-}{-}{-}{-}{-}{-}{-}{-}{-}{-}{-}{-}{-}{-}{-}{-}{-}}
\FunctionTok{png}\NormalTok{(}\FunctionTok{file.path}\NormalTok{(output\_plot\_dir, }\StringTok{\textquotesingle{}out\_of\_state\_tuition\_earnings.png\textquotesingle{}}\NormalTok{))}
\FunctionTok{ggplot}\NormalTok{(college\_scorecard\_df\_analysis, }\FunctionTok{aes}\NormalTok{(}\AttributeTok{x =}\NormalTok{ tuitionfee\_out, }\AttributeTok{y =}\NormalTok{ md\_earn\_wne\_p10, }\AttributeTok{color =} \FunctionTok{as.factor}\NormalTok{(school\_type))) }\SpecialCharTok{+}
  \FunctionTok{geom\_point}\NormalTok{() }\SpecialCharTok{+}
  \FunctionTok{geom\_smooth}\NormalTok{() }\SpecialCharTok{+}
  \FunctionTok{scale\_color\_discrete}\NormalTok{(}\AttributeTok{name =} \StringTok{\textquotesingle{}School Type\textquotesingle{}}\NormalTok{, }\AttributeTok{labels =} \FunctionTok{c}\NormalTok{(}\StringTok{\textquotesingle{}Public\textquotesingle{}}\NormalTok{, }\StringTok{\textquotesingle{}Private\textquotesingle{}}\NormalTok{)) }\SpecialCharTok{+}
  \FunctionTok{xlab}\NormalTok{(}\StringTok{\textquotesingle{}Tuition (Out{-}of{-}State)\textquotesingle{}}\NormalTok{) }\SpecialCharTok{+} \FunctionTok{ylab}\NormalTok{(}\StringTok{\textquotesingle{}Earnings 10 years after completion\textquotesingle{}}\NormalTok{)}
\FunctionTok{dev.off}\NormalTok{()}

\FunctionTok{png}\NormalTok{(}\FunctionTok{file.path}\NormalTok{(output\_plot\_dir, }\StringTok{\textquotesingle{}in\_state\_tuition\_earnings.png\textquotesingle{}}\NormalTok{))}
\FunctionTok{ggplot}\NormalTok{(college\_scorecard\_df\_analysis, }\FunctionTok{aes}\NormalTok{(}\AttributeTok{x =}\NormalTok{ tuitionfee\_in, }\AttributeTok{y =}\NormalTok{ md\_earn\_wne\_p10, }\AttributeTok{color =} \FunctionTok{as.factor}\NormalTok{(school\_type))) }\SpecialCharTok{+}
  \FunctionTok{geom\_point}\NormalTok{() }\SpecialCharTok{+}
  \FunctionTok{geom\_smooth}\NormalTok{() }\SpecialCharTok{+}
  \FunctionTok{scale\_color\_discrete}\NormalTok{(}\AttributeTok{name =} \StringTok{\textquotesingle{}School Type\textquotesingle{}}\NormalTok{, }\AttributeTok{labels =} \FunctionTok{c}\NormalTok{(}\StringTok{\textquotesingle{}Public\textquotesingle{}}\NormalTok{, }\StringTok{\textquotesingle{}Private\textquotesingle{}}\NormalTok{)) }\SpecialCharTok{+}
  \FunctionTok{xlab}\NormalTok{(}\StringTok{\textquotesingle{}Tuition (In{-}State)\textquotesingle{}}\NormalTok{) }\SpecialCharTok{+} \FunctionTok{ylab}\NormalTok{(}\StringTok{\textquotesingle{}Earnings 10 years after completion\textquotesingle{}}\NormalTok{)}
\FunctionTok{dev.off}\NormalTok{()}

\FunctionTok{list.files}\NormalTok{(output\_plot\_dir)}


\FunctionTok{saveRDS}\NormalTok{(college\_scorecard\_df\_analysis, }\AttributeTok{file =} \FunctionTok{file.path}\NormalTok{(output\_file\_dir, }\StringTok{"college\_scorecard\_dataset.RDS"}\NormalTok{))}
\FunctionTok{save}\NormalTok{(college\_scorecard\_df,college\_scorecard\_dict, }\AttributeTok{file =} \FunctionTok{file.path}\NormalTok{(output\_file\_dir, }\StringTok{"college\_scorecard\_objects.RData"}\NormalTok{))}
\FunctionTok{list.files}\NormalTok{(output\_file\_dir)}

\FloatTok{1.}\NormalTok{ While an .RDS file can store only }\DecValTok{1}\NormalTok{ R object, an .RData file can store }\DecValTok{1}\NormalTok{ or more R objects.}
\CommentTok{\# 2. When loading an object from an .RDS file, we need to assign the loaded object to retain it. }
\CommentTok{\#    On the other hand, when loading objects from an .RData file, objects will be loaded directly to your environment without assigning.}

\DocumentationTok{\#\#{-}{-}{-}{-}{-}{-}{-}{-}{-}{-}{-}{-}{-}{-}{-}{-}{-}{-}{-}{-}{-}{-}{-}{-}{-}{-}{-}{-}{-}{-}{-}{-}{-}{-}{-}{-}{-}{-}{-}{-}{-}{-}{-}{-}{-}{-}{-}{-}{-}{-}{-}{-}{-}{-}{-}{-}{-}{-}{-}{-}{-}{-}{-}{-}{-}{-}{-}{-}{-}{-}{-}{-}{-}{-}}
\NormalTok{Plot relationship between }\ControlFlowTok{in}\SpecialCharTok{{-}}\NormalTok{state tuition and median earnings}
\FunctionTok{png}\NormalTok{(}\FunctionTok{file.path}\NormalTok{(output\_plot\_dir, }\StringTok{\textquotesingle{}in\_state\_tuition\_earnings.png\textquotesingle{}}\NormalTok{))}
\FunctionTok{ggplot}\NormalTok{(college\_scorecard\_df\_analysis, }\FunctionTok{aes}\NormalTok{(}\AttributeTok{x =}\NormalTok{ tuitionfee\_in, }\AttributeTok{y =}\NormalTok{ md\_earn\_wne\_p10, }\AttributeTok{color =} \FunctionTok{as.factor}\NormalTok{(school\_type))) }\SpecialCharTok{+}
  \FunctionTok{geom\_point}\NormalTok{() }\SpecialCharTok{+}
  \FunctionTok{geom\_smooth}\NormalTok{() }\SpecialCharTok{+}
  \FunctionTok{scale\_color\_discrete}\NormalTok{(}\AttributeTok{name =} \StringTok{\textquotesingle{}School Type\textquotesingle{}}\NormalTok{, }\AttributeTok{labels =} \FunctionTok{c}\NormalTok{(}\StringTok{\textquotesingle{}Public\textquotesingle{}}\NormalTok{, }\StringTok{\textquotesingle{}Private\textquotesingle{}}\NormalTok{)) }\SpecialCharTok{+}
  \FunctionTok{xlab}\NormalTok{(}\StringTok{\textquotesingle{}Tuition (In{-}State)\textquotesingle{}}\NormalTok{) }\SpecialCharTok{+} \FunctionTok{ylab}\NormalTok{(}\StringTok{\textquotesingle{}Earnings 10 years after completion\textquotesingle{}}\NormalTok{)}
\FunctionTok{dev.off}\NormalTok{()}
\end{Highlighting}
\end{Shaded}

\emph{Note}: These plots above were created using the R library
\texttt{ggplot}. We won't be covering plotting in this course this
quarter, but if you are interested in learning more about it, here are
some resources to get started:

\begin{itemize}
\tightlist
\item
  \href{https://r4ds.had.co.nz/data-visualisation.html}{Data
  visualisation chapter} from \emph{R for Data Science} by Wickham
\item
  \href{https://anyone-can-cook.github.io/rclass2/lectures/ggplot/ggplot.html}{Visualization
  using ggplot2 lecture} from rclass2 last year
\end{itemize}

\textcolor{red}{\textbf{/2}}

\begin{enumerate}
\def\labelenumi{\arabic{enumi}.}
\setcounter{enumi}{4}
\item
  Finally, let's save some of the R objects we've created to flat files:

  \begin{itemize}
  \tightlist
  \item
    Save the \texttt{college\_scorecard\_df\_analysis} dataframe to a
    file called \texttt{college\_scorecard\_dataset.RDS} in your
    \texttt{output\_file\_dir}
  \item
    Save the \texttt{college\_scorecard\_df} and
    \texttt{college\_scorecard\_dict} objects to a file called
    \texttt{college\_scorecard\_objects.RData} in your
    \texttt{output\_file\_dir}
  \item
    List the contents of \texttt{output\_file\_dir} and copy the output
    as a comment in your script
  \item
    Also add a comment in your script to answer the following question:
    What are the 2 main differences between an \texttt{.RDS} and
    \texttt{.RData} file (in terms of saving and loading the data)?
  \end{itemize}
\end{enumerate}

\hypertarget{part-iv-create-a-github-issue}{%
\subsection{Part IV: Create a GitHub
issue}\label{part-iv-create-a-github-issue}}

\begin{itemize}
\item
  Go to the
  \href{https://github.com/anyone-can-cook/rclass2_student_issues_w23/issues}{class
  repository} and create a new issue.
\item
  Please refer to
  \href{https://github.com/anyone-can-cook/rclass2_student_issues_w23/blob/main/README.md}{rclass2
  student issues readme} for instructions on how to post questions or
  things you've learned.
\item
  You can either:

  \begin{itemize}
  \tightlist
  \item
    Ask a question that you have about this problem set or the course in
    general. Make sure to assign the instructors (@ozanj,
    @xochilthlopez, @joycehnguy, @augias) and mention your team (e.g.,
    @anyone-can-cook/your\_team\_name).
  \item
    Share something you learned from this problem set or the course.
    Please mention your team (e.g., @anyone-can-cook/your\_team\_name).
  \end{itemize}
\item
  You are also required to respond to at least one issue posted by
  another student.
\item
  Paste the url to your issue here:
\item
  Paste the url to the issue you responded to here:
\end{itemize}

\hypertarget{submit-problem-set}{%
\section{Submit problem set}\label{submit-problem-set}}

\begin{itemize}
\tightlist
\item
  Go to the \href{https://anyone-can-cook.github.io/rclass2/}{class
  website} and under the ``Readings \& Assignments''
  \textgreater\textgreater{} ``Week 1'' tab, click on the ``Problem set
  1 submission link''
\item
  Submit your \texttt{lastname\_firstname\_ps1.R} file
\item
  No need to submit any other files that were created
\end{itemize}

\end{document}
